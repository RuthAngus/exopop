\documentclass[12pt,preprint]{aastex}

\usepackage{color,hyperref}
\definecolor{linkcolor}{rgb}{0,0,0.5}
\hypersetup{colorlinks=true,linkcolor=linkcolor,citecolor=linkcolor,
            filecolor=linkcolor,urlcolor=linkcolor}
\usepackage{url}
\usepackage{amssymb,amsmath}

\newcommand{\project}[1]{{\sffamily #1}}
\newcommand{\emcee}{\project{emcee}}
\newcommand{\kepler}{\project{Kepler}}
\newcommand{\license}{MIT License}

\newcommand{\paper}{\emph{Article}}

\newcommand{\foreign}[1]{\emph{#1}}
\newcommand{\etal}{\foreign{et\,al.}}
\newcommand{\etc}{\foreign{etc.}}

\newcommand{\Fig}[1]{Figure~\ref{fig:#1}}
\newcommand{\fig}[1]{\Fig{#1}}
\newcommand{\figlabel}[1]{\label{fig:#1}}
\newcommand{\Tab}[1]{Table~\ref{tab:#1}}
\newcommand{\tab}[1]{\Tab{#1}}
\newcommand{\tablabel}[1]{\label{tab:#1}}
\newcommand{\Eq}[1]{Equation~(\ref{eq:#1})}
\newcommand{\eq}[1]{\Eq{#1}}
\newcommand{\eqlabel}[1]{\label{eq:#1}}
\newcommand{\Sect}[1]{Section~\ref{sect:#1}}
\newcommand{\sect}[1]{\Sect{#1}}
\newcommand{\App}[1]{Appendix~\ref{sect:#1}}
\newcommand{\app}[1]{\App{#1}}
\newcommand{\sectlabel}[1]{\label{sect:#1}}

\newcommand{\dd}{\ensuremath{\,\mathrm{d}}}
\newcommand{\bvec}[1]{\ensuremath{\boldsymbol{#1}}}

% TO DOS
\newcommand{\todo}[3]{{\color{#2} \emph{#1} TODO: #3}}
\newcommand{\dfmtodo}[1]{\todo{DFM}{red}{#1}}
\newcommand{\hoggtodo}[1]{\todo{HOGG}{blue}{#1}}
\newcommand{\mortontodo}[1]{\todo{MORTON}{green}{#1}}

% Document specific variables.
\newcommand{\rate}{\ensuremath{\eta}}
\newcommand{\ratepars}{\ensuremath{\theta}}
\newcommand{\obs}[1]{\ensuremath{\hat{#1}}}
\newcommand{\radius}{\ensuremath{r}}
\newcommand{\period}{\ensuremath{T}}
\newcommand{\completeness}{\ensuremath{P_\mathrm{c}}}
\newcommand{\transitprob}{\ensuremath{P_\mathrm{t}}}

\newcommand{\interim}{\ensuremath{\alpha}}

\begin{document}

\title{%
    Inferring the rate of Earth-like exoplanets from noisy data
}

\newcommand{\nyu}{2}
\newcommand{\mpia}{3}
\newcommand{\princeton}{4}
\author{%
    Daniel~Foreman-Mackey\altaffilmark{1,\nyu},
    David~W.~Hogg\altaffilmark{\nyu,\mpia},
    Timothy~D.~Morton\altaffilmark{\princeton},
    \etal
}
\altaffiltext{1}            {To whom correspondence should be addressed:
                             \url{danfm@nyu.edu}}
\altaffiltext{\nyu}         {Center for Cosmology and Particle Physics,
                             Department of Physics, New York University,
                             4 Washington Place, New York, NY, 10003, USA}
\altaffiltext{\mpia}        {Max-Planck-Institut f\"ur Astronomie,
                             K\"onigstuhl 17, D-69117 Heidelberg, Germany}
\altaffiltext{\princeton}   {Department of Astrophysics, Princeton University,
                             Princeton, NJ}

\begin{abstract}

We infer the distribution and rate of Earth-like exoplanets around Sun-like
stars.

\end{abstract}

\keywords{%
exoplanets: sickness
---
code: open-source
---
keywords: made-up-by-Hogg
}

\section{Introduction}

\section{The model}

\emph{Note: I think that we should use $T$ instead of $P$ for period because
$P$ is too heavily overloaded by the probabilities\ldots}

We have some parametric function \rate\ describing the true rate of exoplanets
with periods \period\ and radii \radius
\begin{eqnarray}\eqlabel{true-rate}
\rate (\period,\,\radius;\,\ratepars)
&=& \frac{\dd N}{\dd\ln\period\dd\ln\radius}\quad.
\end{eqnarray}
Here we have not made any assumptions about the functional form of \rate, we
are only stating that it can be modeled by some---possibly very large---set of
parameters \ratepars.

In this \paper, we will assume that the completeness function of the survey is
known and it can be represented as a probability at every period and radius
$\completeness (\period,\,\radius)$.
Similarly, we can consider the marginalized transit probability $\transitprob
(\period)$ as only a function of period.
These selection effects can be incorporated into the rate from \eq{true-rate}
to get the \emph{observed rate}
\begin{eqnarray}\eqlabel{obs-rate}
\obs{\rate} (\period,\,\radius;\,\ratepars)
&=& \rate (\period,\,\radius;\,\ratepars)\,\completeness(\period,\,\radius)\,
\transitprob(\period)\quad.
\end{eqnarray}

Now we will make the further assumption that every candidate in our sample of
detected exoplanets is an independent Poisson sample from the population
described by our rate function.
Under this assumption, the likelihood of the catalog conditioned on a set of
parameters \ratepars\ is
\begin{eqnarray}
\ln p(\period,\,\radius\,|\,\ratepars) &\propto&
\sum_{k=1}^K \ln\obs{\rate} (\period_k,\,\radius_k;\,\ratepars)
- \int \obs{\rate} (\period,\,\radius;\,\ratepars) \dd\period\dd\radius
\quad.
\end{eqnarray}

\hoggtodo{How is this related to the version of the likelihood that doesn't
include the normalization. What is the best way to introduce both forms?}

\section{Including observational uncertainties}

In the previous section, we implicitly assumed that the measurements of
\period\ and \radius\ given by the catalog were perfect, without any
observational uncertainties.
This isn't a terrible approximation for the period measurements but not for
the radii.
Luckily, it is not hard to include these effects in our probabilistic
framework.

In general, catalogs produce posterior probability distributions for
quantities conditioned on data.
In particular, the catalog that we're considering here provides estimates of
the posterior distribution of \period\ and \radius\ conditioned on the
observables $\obs{\period}$ and $\obs{\radius}$ under some \emph{interim
priors} parameterized by \interim
\begin{eqnarray}
p(\period,\,\radius\,|\,\obs{\period},\,\obs{\radius},\,\interim) \quad.
\end{eqnarray}
This estimate can be propagated as \emph{posterior samples}
\begin{eqnarray}
\{\period^{(n)},\,\radius^{(n)}\} &\sim&
p(\period,\,\radius\,|\,\obs{\period},\,\obs{\radius},\,\interim) \quad.
\end{eqnarray}

Now the quantity that we need to compute is again the (marginalized)
likelihood of the data $\obs{\period}$ and $\obs{\radius}$ conditioned on a
set of rate function parameters \ratepars
\begin{eqnarray}
p(\obs{\period},\,\obs{\radius}\,|\,\ratepars) &=&
\int p(\period,\,\radius,\,\obs{\period},\,\obs{\radius}\,|\,\ratepars)
\dd\period\dd\radius \\
&=&
\int p(\period,\,\radius\,|\,\ratepars) \,
p(\obs{\period},\,\obs{\radius}\,|\,\period,\,\radius)
\dd\period\dd\radius \\
&=&
\int p(\period,\,\radius\,|\,\ratepars) \,
p(\obs{\period},\,\obs{\radius}\,|\,\period,\,\radius)\,
\frac{p(\period,\,\radius\,|\,\obs{\period},\,\obs{\radius},\,\interim)}
     {p(\period,\,\radius\,|\,\obs{\period},\,\obs{\radius},\,\interim)}
\dd\period\dd\radius \\
&=&
p(\obs{\period},\,\obs{\radius}\,|\,\interim)\,
\int
\frac{p(\period,\,\radius\,|\,\ratepars) \,
      p(\obs{\period},\,\obs{\radius}\,|\,\period,\,\radius)}
     {p(\period,\,\radius\,|\,\interim) \,
      p(\obs{\period},\,\obs{\radius}\,|\,\period,\,\radius)}\,
p(\period,\,\radius\,|\,\obs{\period},\,\obs{\radius},\,\interim)
\dd\period\dd\radius \\
\frac{p(\obs{\period},\,\obs{\radius}\,|\,\ratepars)}
     {p(\obs{\period},\,\obs{\radius}\,|\,\interim)}
&=&
\int
\frac{p(\period,\,\radius\,|\,\ratepars)}
     {p(\period,\,\radius\,|\,\interim)} \,
p(\period,\,\radius\,|\,\obs{\period},\,\obs{\radius},\,\interim)
\dd\period\dd\radius \\
&\approx&
\frac{1}{N}
\sum_{n=1}^N
\frac{p(\period^{(n)},\,\radius^{(n)}\,|\,\ratepars)}
     {p(\period^{(n)},\,\radius^{(n)}\,|\,\interim)}
\end{eqnarray}

\acknowledgments
It is a pleasure to thank
    Erik Petigura (Berkeley),
    \ldots
for helpful contributions to the ideas and code presented here.
This project was partially supported by the NSF (grant AST-0908357), and NASA
(grant NNX08AJ48G).

\newcommand{\arxiv}[1]{\href{http://arxiv.org/abs/#1}{arXiv:#1}}
\begin{thebibliography}{}\raggedright

\bibitem[Petigura \etal(2013)]{petigura}
Petigura, E.~A., Howard, A.~W., \& Marcy, G.~W.\ 2013,
Proceedings of the National Academy of Science, 110, 19273

\end{thebibliography}

\end{document}
